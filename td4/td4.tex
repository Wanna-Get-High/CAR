\documentclass[a4paper,10pt]{article}
\include{amsymb}
\include{marvosym}

\usepackage[francais]{babel}
\usepackage[cyr]{aeguill}
\usepackage[applemac]{inputenc}
\usepackage{graphicx}
\usepackage{xspace}
\usepackage[a4paper]{geometry}
\usepackage{latexsym,amsmath,amssymb,textcomp}
\usepackage{moreverb}
\usepackage{listings}
\usepackage{multirow}
\usepackage{titling}
\usepackage{mathabx}

\usepackage{pdfpages}

\usepackage{hyperref}


\newlength{\indentationnota}
\newlength{\largeurlignenota}
\newlength{\paddingnota}
\newlength{\largeurnota}
\setlength{\paddingnota}{5pt}
\setlength{\largeurnota}{0.9cm}

\definecolor{pink}{rgb}{1,0.5,1}
\makeatletter

\newenvironment{pictonote}[1]{% on passe le nom du �chier en argument
\begin{list}{}{%
\setlength{\labelsep}{0pt}%
\setlength{\rightmargin}{15pt}%
\setlength{\paddingnota}{5pt}}
\item%
\setlength{\indentationnota}{%
\@totalleftmargin+\largeurnota+\paddingnota}%
\setlength{\largeurlignenota}{%
\linewidth-\largeurnota-\paddingnota}%
\parshape=3%
\indentationnota\largeurlignenota%
\indentationnota\largeurlignenota%
\@totalleftmargin\linewidth%
\raisebox{-\largeurnota+2.2ex}[0pt][0pt]{%
\makebox[0pt][r]{%
\includegraphics[width=\largeurnota]{#1}%
\hspace{\paddingnota}}}%
\ignorespaces}{%
\end{list}}

\makeatother

\newenvironment{attention}%
{\begin{pictonote}{/Users/benjamin/Documents/Education/LaTeX/danger}}%
{\end{pictonote}}

\newenvironment{unicorn}%
{\begin{pictonote}{/Users/benjamin/Documents/Education/LaTeX/angry_unicorn}}%
{\end{pictonote}}

\newcommand\unicornbox[1]{\colorbox{pink}{\color{white}{#1}}}

\newcommand\cf{\emph{c.f}}

\newcommand\signature{%
\begin{figure}[br]
	\begin{flushright}
	\begin{minipage}{8cm}
		\begin{center}
		\reflectbox{\includegraphics[width=8cm]{/Users/WannaGetHigh/workspace/latex/unicorn.png}}\\
		\theauthor
		\end{center}
	\end{minipage}
	\end{flushright}
\end{figure}}

\newcommand\danger{\raisebox{-0.4ex}{\LARGE $\triangle$ \normalsize} \hspace{-4.1ex}! \hspace{1ex}}
\newcommand\PL{Programmation Logique\xspace}
\newcommand\pl{\bsc{Prolog}\xspace}


\title{CAR - S�ance 4 : Repr�sentation des donn�es}
\author{Benjamin \bsc{Van Ryseghem}}


\begin{document}
\maketitle

\section{Petit boutiste et grand boutiste}

\subsection{Question 1}
\verb?"Hello World"?

\paragraph{Diff�rents codages:}

\mbox{}

\begin{tabular} { ll}
Machine A (big endian) : & 000B 48 65 6C 6C 6F 20 57 6F 72 6C 64 \\
Machine B (little endian) : & OB00 65 48 6C 6C 20 6F 6F 57 6C 72 64 \\
\end{tabular}

\subsection{Question 2}

Du caca !

\subsection{Question 3}

OB 00 65 48 6C 6C 20 6F 6F 57 6C 72 64

\paragraph{Conclusion:}
Il faut se mettre d'accord sur le format � utiliser (bas niveau)

\section{M�canisme d�encodage des donn�es en Java}

\subsection{Question 1}
\verb?AC ED 00 05 77 08 00 00 00 11 00 00 00 02?

\mbox{}

\begin{tabular}{|c|c|}
\hline
0xACED & Magic stream \\
0x0005 & Stream version \\
0x77 & Block Data \\
0x00000011 & 17 \\
0x00000002 & 2 \\
\hline
\end{tabular}

\subsection{Question 2}
\verb?AC ED 00 05 77 07 00 05 48 65 6C 6C 6F?

\mbox{}

\begin{tabular}{|c|c|}
\hline
0xACED & Magic stream \\
0x0005 & Stream version \\
0x77 & Block Data \\
0x07 & taille totale de la stream (7 octets) \\
0x0005 & taille de la string (5 caract�res) \\
0x48 & \$h \\
0x65 & \$e \\
0x6C & \$l \\
0x6C & \$l \\
0x6F & \$o \\
\hline
\end{tabular}

\subsection{Question 3}

\begin{tabular}{|c || c|}
\hline
Avantages & Inconv�nients\\
\hline
\hline
foo & bar\\
\hline
\end{tabular}

\subsection{Question 4}


\subsection{"hello world" }

Big indian: 000B 6865 6C6C 6F20 776F 726C 64

Little indian: 0B00 6865 6C6C 6F20 776F 726C 64

\subsection{Que se passe-t-il si la machine A envoie la cha�ne de caract�res telle quelle � la machine B ?}

Il ne vont pas se comprendre ... PB pour la taille.

\subsection{Dans un deuxi�me temps, on d�cide d�inverser l�ordre des octets pour chaque couple d�octets arrivant sur B. Que se passe-t-il ?}

0B00 6568 6C6C 206F 776F 6C72 0064

message = "ehllowlrd"

\section{M�canisme d�encodage des donn�es en Java}

\subsection{D�coder la s�quence d�octets suivante sachant que 2 entiers (int) ont �t� �crits dans le flux. Quelle est la valeur de ces entiers ?}

\begin{verbatim}
Pour  AC ED 00 05 77 08 00 00 00 11 00 00 00 02

STREAM_MAGIC =  AC ED
STREAM_VERSION = 00 05
TC_BLOCKDATA = 77
int = 00 00 00 11 = 17
int = 00 00 00 02 = 2
\end{verbatim}

\subsection{D�coder la s�quence d�octets suivante sachant qu�une cha�ne de caract�res a �t� �crite dans le flux.}

\begin{verbatim}
Pour AC ED 00 05 77 07 00 05 48 65 6C 6C 6F

STREAM_MAGIC  = AC ED
STREAM_VERSION = 00 05
TC_BLOCKDATA = 77
7 octet pour la string = 07
2 octet pour la taille = 00 05
STRING = 48 65 6C 6C 6F  =  "hello"
\end{verbatim}

\subsection{Quels en sont les avantages et les inconv�nients ?}


\subsection{Coder un objet de la classe Point2D suivante ayant pour valeur de x 18 et pour valeur de y 20}

\begin{verbatimtab}
class Point2D implements Serializable { 
	private long x;
	private long y; 
}
\end{verbatimtab}

\begin{verbatimtab}
ACED				magic
0005		 		streamVersion
73				TC_OBJECT
72				TC_CLASSDESC
0007				taille string
50 6F 69 6E 74 32		"Point2D"
00 00 00 00 00 00 00		SUID
00 00 00 00 00 00 00		new handle
02 00 00 00 00 00 00		classDescFlag (serializable)
00 02 00 00 00 00 00		count 2 champ
4A				long
0001				taille string
78				x
4A				long
0001				taille string
79				y
78				TC_END_BLOCK_DATA
70				TC_NULL
01				new Handle (1)
00 00 00 00 00 00 00 12	18 	val du premier champ
00 00 00 00 00 00 00 14	20 	val du deuxi�me champ
\end{verbatimtab}
\signature

\end{document}