 \documentclass[a4paper,10pt]{article}
\input{/Users/WannaGetHigh/workspace/latex/macros.tex}

\title{CAR - TD4 : Repr�sentation des donn�es}
\author{Fran�ois \bsc{Lepan}}

\begin{document}
\maketitle

\section{Petit boutiste et grand boutiste}
\subsection{"hello world" }

Big indian: 000B 6865 6C6C 6F20 776F 726C 64

Little indian: 0B00 6865 6C6C 6F20 776F 726C 64

\subsection{Que se passe-t-il si la machine A envoie la cha�ne de caract�res telle quelle � la machine B ?}

Il ne vont pas se comprendre ... PB pour la taille.

\subsection{Dans un deuxi�me temps, on d�cide d�inverser l�ordre des octets pour chaque couple d�octets arrivant sur B. Que se passe-t-il ?}

0B00 6568 6C6C 206F 776F 6C72 0064

message = "ehllowlrd"

\section{M�canisme d�encodage des donn�es en Java}

\subsection{D�coder la s�quence d�octets suivante sachant que 2 entiers (int) ont �t� �crits dans le flux. Quelle est la valeur de ces entiers ?}

\begin{verbatim}
Pour  AC ED 00 05 77 08 00 00 00 11 00 00 00 02

STREAM_MAGIC =  AC ED
STREAM_VERSION = 00 05
TC_BLOCKDATA = 77
int = 00 00 00 11 = 17
int = 00 00 00 02 = 2
\end{verbatim}

\subsection{D�coder la s�quence d�octets suivante sachant qu�une cha�ne de caract�res a �t� �crite dans le flux.}

\begin{verbatim}
Pour AC ED 00 05 77 07 00 05 48 65 6C 6C 6F

STREAM_MAGIC  = AC ED
STREAM_VERSION = 00 05
TC_BLOCKDATA = 77
7 octet pour la string = 07
2 octet pour la taille = 00 05
STRING = 48 65 6C 6C 6F  =  "hello"
\end{verbatim}

\subsection{Quels en sont les avantages et les inconv�nients ?}


\subsection{Coder un objet de la classe Point2D suivante ayant pour valeur de x 18 et pour valeur de y 20}

\begin{verbatimtab}
class Point2D implements Serializable { 
	private long x;
	private long y; 
}

ACED				magic
0005		 			streamVersion
73					TC_OBJECT
72					TC_CLASSDESC
0007					taille string
50 6F 69 6E 74 32		"Point2D"
00 00 00 00 00 00 00	SUID
00 00 00 00 00 00 00	new handle
02 00 00 00 00 00 00	classDescFlag (serializable)
00 02 00 00 00 00 00	count 2 champ
4A					long
0001					taille string
78					x
4A					long
0001					taille string
79					y
78					TC_END_BLOCK_DATA
70					TC_NULL
01					new Handle (1)
00 00 00 00 00 00 00 12	18 val du premier champ
00 00 00 00 00 00 00 14	20 val du deuxi�me champ






\end{verbatimtab}













\end{document}