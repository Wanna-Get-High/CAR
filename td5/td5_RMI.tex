 \documentclass[a4paper,10pt]{article}
\input{/Users/WannaGetHigh/workspace/latex/macros.tex}

\title{TD 5/TD 6 � RMI}
\author{Fran�ois \bsc{Lepan}}

\begin{document}
\maketitle

\section{}

\section{Election sur un anneau d'objets RMI}
\subsection{Quel est le test d'arr�t de l'�lection i.e. du  "tour de l'anneau" ? }

\subsection{Quelle information faut-il inclure dans le message qui fait le "tour de l'anneau" pour pouvoir r�aliser ce test ?}

\subsection{Repr�senter sur un diagramme les �changes de messages engendr�s par l�ex�cution d�une �lection avec l�objet 2 comme initiateur}

\subsection{Quel est le test de d�cision de l'�lu ?}

\subsection{Quelle information faut-il inclure dans le message qui fait le " tour de l'anneau" pour pouvoir r�aliser ce test ?}

\subsection{� la fin d'un tour, quel(s) objet(s) conna�(ssen)t l'identifiant de l'�lu ? Pourquoi ?}

\subsection{Pour chacune des solutions propos�es � la question pr�c�dente, d�finir l'interface de la classe Java correspondant � un objet de l'anneau}

\begin{verbatimtab}
public interface NoeudElection extends Remote {
	//  Bleu 
	// On ajoute l'initiateur si il y a plusieurs vague lancer en mm temps 
	// afin de ne pas avoir de concurrence entre les vagues
	// appelant -> pour savoir de quel cote propager
	public void electionChef(NoeudElection appelant, NoeudElection chef) throws RemoteException;
	
	// Rouge
	public void propagationChef(NoeudElection appelant, NoeudElection chef) throws RemoteException;
	
	// en passant la valeur de puissance dans la 1ere fonction on peut enlever cette fction -> optimisation
	int getPuissance();
}
\end{verbatimtab}

\subsection{Pour chacune des solutions propos�es � la question pr�c�dente, donner en Java l'implanta- tion de la m�thode election}

\begin{verbatimtab}

publci Noeud implements NoeudElection {

	// cette valeur est suppos�e initialis� par un joli Math.random()
	private int myVal
	
	private NoeudElection chef;
	private NoeudElection gauche;
	private NoeudElection droite;
	
	public void electionChef(NoeudElection appelant, NoeudElection chef, int chefVal) throws RemoteException {
		
		// Bleu
		if (chef == null) {
			if (this.myVal > chefVal) {
				this.chef = this;
			} else {
				this.chef = chef;
				this.myVal = chefVal;
			}
			
			if (this.gauche == appelant) {
				this.droite.election(this,this.chef,this.myVal)
			} else {
				this.gauche.election(this,this.chef,this.myVal)			
			}
		// Rouge	
		} else {
			if (this.myVal < chefVal) {
				this.myVal = chefVal;
				this.chef = chef;
			}
			
			appelant.propagationChef(this,this.chef);
		}
	}
}
\end{verbatimtab}

\end{document}