 \documentclass[a4paper,10pt]{article}
\input{/Users/WannaGetHigh/workspace/latex/macros.tex}

\title{TD: 7 - 8 RMI}
\author{Fran�ois \bsc{Lepan}}

\begin{document}
\maketitle

\section{Annuaire}

\section{Instanciation � la demande}

\subsection{Ecrire le code Java de la classe \emph{ClientPingThread}}

\begin{verbatimtab}

public interface CalculIrf extends Remote {
	public Serialisable submit(String arg) throws RemoteException;
	public String getHost() throws RemoteException;
	public int getPort() throws RemoteException;
}

public class ClientPingThread exends Thread {
	
	Socket s;
	DataInputStream is;
	DataOutPutStream os;
	public static final int TEN_SECONDS = 10 000;
	
	public ClientPingThread(CalculIrf computer) {
		this.s = new Socket(computer.getHost(),computer.getPort());
		this.os = new DataInputStream(s.getOutPutStream());
		this.is = new DataOutputStream(s.getInputStream());
		this.s.setTimeout(TEN_SECONDS);
	}
	
	public void run() throws RMIException {
		int nbEssais = 3;
		while(nbEssais-- > 0) {
			if (ping()) nbEssais = 3;
		}
		throw new RMIException();
	}
	
	public boolean ping() throws RMIException {
		this.s.send("PING");
		
		try {
			os.writeUTF("PING");
			return "ACK".equals(is.readUTF());
			
		} catch(IOTimeout) {
			return false
		}
	}
}

public MAIN {
	calculIrf
}
\end{verbatimtab}

\subsection{Est-il n�cessaire que les objets serveurs RMI cr��s par le serveur Web soient enregistr�s dans le service de noms de RMI (rmiregistry) ?}

pas besoin de l'enregistrer car on a CAlculeIrF qui permet de savoir a quel objet on communique.


\subsection{Si l�on veut que le serveur Web puisse retourner la m�me r�f�rence d�objets RMI � deux clients s�ex�cutant simultan�ment, de quoi faut-il s�assurer ?}

Qu'il soit Thread safe -> synchronize

\subsection{D�crire en fran�ais l�algorithme de fonctionnement du serveur Web}

On nous demande un objet 
	-> si cr�er -> on v�rifie si Threa safe -> on le renvoi.
	-> si l'objet n'est pas cr�er -> on en cr�er un qui est ThreadSafe -> on le renvoi

\section{Patron Observateur/Sujet}

\subsection{�crire l'interface Java RMI des deux objets Observer et Subject.}
\begin{verbatimtab}
	public interface RMIObserver extends Remote {
		void notify() throws RemoteException;
	}
	
	public interface RMIObservable extends Remote {
		void addObserver(RMIObserver) {
		
		}
	}
\end{verbatimtab}

\end{document}




